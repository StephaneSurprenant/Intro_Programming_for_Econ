\documentclass[12pt]{article}
% Cette partie sert à définir un format pour le
% document Latex. Notez qu'il y a une option pour
% écrire recto-verso (two sided). Vous pouvez 
% facilement trouver l'information sur ceci en ligne.

\usepackage[utf8]{inputenc}
\usepackage[french]{babel}
% Ces packages donnent accès à des fonctionalités
% utiles pour la rédaction de documents français. Les
% options sont indiquées entre corchets, avant le nom
% du package. Sans ces packages, il vous faudrait coder
% les accents à la main: il faudrait écrire \'{a} plutôt que à.

\usepackage{amsmath, amssymb, amsfonts}
% Le premier package donne accès à des commandes et des
% environnements mathématiques tel que "align" qui
% permet d'aligner plusieurs équations. Les deux autres
% donnent accès à des symboles et polices utiles pour
% les maths.

\usepackage{hyperref}
% Ceci permet de créer des hyperliens entre différents
% éléments du texte comme par exemple une référence
% comme Stock et Watson (2002a) et la référence complète
% dans la bibliographie ou encore des numéros de tables,
% de figures ou d'équations et les tables, figures et
% équations auxquels ils réfèrent.

\usepackage[dvipsnames]{xcolor}
% Pour ajouter les couleurs pour hyperref.

\usepackage{graphicx}
% Ce package permet l'ajout d'options à la commande
% \includegraphics pour l'inclusion de figures.

\usepackage{subcaption}
% Pour les figures composées.

\usepackage{natbib}
% Pour les commandes de citations.

\usepackage{booktabs}
% Contient les fonctions \toprule, \midrule et \bottomrule.

\usepackage{float, morefloats}
% Contient les options pour contrôler la disposition
% de figures et de tableaux.

\hypersetup{
	colorlinks,
	citecolor=Green,
	linkcolor=red,
	urlcolor=blue}
% Les options pour les hyperliens.

\newcommand{\pd}[2]{\frac{\partial #1}{\partial #2}}
% Commande personalisée pour faciliter la création de
% dérivées partielles.

\newcommand{\argmin}[1]{\arg\underset{#1}{\min}}
% Pour "argmin".

\newcommand{\CustomTitlePage}[8]{
	\begin{titlepage}
		\begin{center}
			
			UNIVERSITÉ DU QUÉBEC À MONTRÉAL \\
			\vspace{8em}
			
			#1 \\ %TITRE DU TRAVAIL
			\vspace{8em}
			
			TRAVAIL PRÉSENTÉ À \\
			#2 \\ % NOM DU PROFESSEUR(E)
			\vspace{1em}
			DANS LE CADRE DU COURS \\
			#3 \\ % NOM DU COURS
			#4, gr. #5 \\ % SIGLE DU COURS, gr. GROUPE
			\vspace{8em}
			
			PAR \\ %% \\ saute à l'autre ligne
			#6 \\
			#7 \\
			\vspace{8em}
			
			#8
			
		\end{center}
	\end{titlepage}
	
}

\begin{document}
	
	\CustomTitlePage{TITRE DU TRAVAIL}{NOM DU/DE LA PROFESSEUR(E)}{NOM DU COURS}{SIGLE}{No DE GROUPE}{VOTRE NOM}{VOTRE MATRICULE}{\today}
		
	\begin{align} % align*: pour enlever tous les chiffres
		\hat{\beta} &:= \argmin{\beta \in \mathbb{R}^K} \left\{ || Y - X\beta ||_2^2 \right\} \label{definition} \\
		            &= \argmin{\beta \in \mathbb{R}^K} \left\{ (Y - X\beta)'(Y - X\beta) \right\} \notag \\ 
		\rightarrow & \hat{\beta} = (X'X)^{-1}X'Y \label{solution}
	\end{align}   % align*: il faut mettre le même dans \begin et \end
	
	Voici les équations (\ref{definition}) and (\ref{solution}). Le tableau (\ref{table}) est ci-bas, tout comme la figure (\ref{figure}).
	
	\begin{table}[H] % H force le positionnement dans le texte.
		\begin{center}
			\caption{Exemple} \label{table} % Les étiquette permettent les
			\begin{tabular}{l|cc}           % références croisées dynamiques.
				\toprule \toprule
				& \multicolumn{2}{c}{Exemple} \\
				& Col. Title 1 & Col. Title \\ \midrule
				R. Title 1 & 1$^{**}$ & 2 \\
				R. Title 2 & 3 & 4 \\ \bottomrule \bottomrule
			\end{tabular}
		\end{center}
		\begin{footnotesize}
			\flushleft         % Ceci force un ajustement à gauche
			Note: Un exemple.  % dans l'environnement où il se trouve.
		\end{footnotesize}
	\end{table}
	
	\begin{figure}[H] % H force le positionnement dans le texte.
		\begin{center}
			\begin{subfigure}{.5\textwidth} % La moitié de la largeur du texte
				\includegraphics[width=\textwidth]{../googleScholar.png}
				\subcaption{A}
			\end{subfigure}%
			\begin{subfigure}{.5\textwidth}
				\includegraphics[width=\textwidth]{../googleScholar.png}
				\subcaption{B}
			\end{subfigure}
			\caption{Un exemple de figure} \label{figure}
		\end{center}
		\begin{footnotesize}
			\flushleft
			Note: Un exemple.
		\end{footnotesize}
	\end{figure}
	
	% Notes: Sans le commentaire entre les subfigures, elles sont empilées
	% verticalement. Aussi, souvenez-vous: ../ veut dire <<remonter>> d'un
	% niveau, ../../ <<remonter>> de deux niveaux, etc.
	
	\cite{cit1} % Il y aussi \citep pour entre parenthèse. 
	
	\clearpage                              % Force le passage à une autre page
	\renewcommand{\refname}{Bibliographie}  % Change le titre de section
	\bibliographystyle{../apalike-uqam.bst} % Style UQAM de bibliographie
	\bibliography{../bibliographie}         % Importer le fichier .bib
	
\end{document}